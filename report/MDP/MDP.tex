\graphicspath{{MDP/fig/}}

\chapter{Markov Decision Processes}
\label{chap:MDP}
\section{Introduction}

In order to describe reinforcement learning techniques, we first have to have a way of mathematically defining reinforcement learning problems. In this chapter, we do so by formally introduce the concept of Markov decision processes (MDPs). This chapter introduces the key concepts required to address the reinforcement learning problem. These concepts includes, state and action space, returns, value functions and Bellman equations.
The majority of the theory discussed in this chapter is based on the work by Sutton and Barto \cite{sutton_barto}.
\section{Definition of a MDP}
A Markov decision process is a stochastic, discrete, time controlled process. What this means is that a MDP is a partially controlled and partially random process \cite{sutton_barto}.
In general RL problems can be described using MDPs \cite{David_Silver}. 
Paul Weng states that, MDP's are described by the following: state-space S, action-space A, state transition probability $P^{a}_{ss'}$, reward function $R^{a}_{s}$ and discount factor $\gamma$\cite{MDP_Paul_Weng}. These variables can be denoted in a 5-tuple as
\begin{equation}
	(S,A,P^{a}_{ss'},R^{a}_{s},\gamma)
\end{equation}
 where,
\begin{itemize}
	\item S is a finite set of states with state s $\in$ S
	\item A is a finite set of actions with action a $\in$ A
	\item $P^{a}_{ss'}$ is a matrix of probabilities which predicts the next state s', where\\ $P^{a}_{ss'} = P(S_{t+1} = s' | S_t = s,A_t = a) $
	\item $R^{a}_{s}$ is the immediate reward after transitioning from state s to s' using action a, where $R^{a}_{s} = E[R_{t+1}|S_t =s, A_t =a]$
	\item $\gamma$ $\in$ [0,1] is the discount factor applied to the reward
\end{itemize}
In an MDP, the assumption is that all the states have the Markov property \cite{sutton_barto}. This means that
\begin{equation}
	P[S_{t+1}|S_{t}] = P[S_{t+1}|S_{1},...,S_{t}].
\end{equation}
In other words, the probability of transitioning to state $S_{t+1}$ given state $S_t$, must be equal to the probability of transitioning to state $S_{t+1}$ given all states $S_1$ to $S_t$. This means that the current state is required to contain all the information of the previous states.

\section{Value function and policies}
The return $G_t$ is the total sum of the reward at each time step, multiplied by a constant diminishing factor $\gamma$.
The discount $\gamma$ $\in$ [0,1] determines the present value of future reward $R_{t}$.

\begin{align}
	G_t &= R_{t+1}+\gamma R_{t+2} +\gamma^2 R_{t+3}+ ... \nonumber\\
	&= \sum_{k=0}^{\infty}\gamma^{k} R_{t+k+1}
	\label{eq:G_t}
\end{align}

For $\gamma=0$, the return $G_t$ only depends on the single reward that can be obtained in the next step, namely $R_{t+1}$. When $\gamma=0$ the return $G_t$ is 'short-sighted'. For $\gamma=1$ on the other hand, the return depends on all the rewards that are projected to be obtained until the process terminates. $G_t$ is thus "far-sighted" for $\gamma=1$ \cite{sutton_barto}.
What can also note is that using a discount factor between 0 and 1 makes the return finite because $R_{t+k+1}$ is finite and
\begin{equation}
	\sum_{k=0}^{\infty}\gamma^{k}=\frac{1}{1-\gamma}\hspace{5pt},
\end{equation}
which is finite for $\gamma$ $\in$ [0,1]. \\
The probability that the agent will choose a specific action given the state is called its policy $\pi(a|s)$. Mathematically this is defined as
\begin{equation}
	\pi(a|s) = P[A_t = a, S_t = s].
\end{equation}
Another important definition is the value function v(s) defined as
\begin{equation}
	v(s) = E[G_t | S_t = s].
\end{equation}
This equation shows that the value function v(s) is the expected return given that v(s) is evaluated in state s at time-step t.
\section{Bellman Expectation Equation}
Sutton and Barto \cite{sutton_barto} decomposes v(s) so that it becomes a recursive function as follows
\begin{align}
	v(s) &= E[G_t | S_t = s]\\
	&= E[R_{t+1} + \gamma R_{t+2} + \gamma^{2} R_{t+3} + ...|S_t = s]\\
	&= E[R_{t+1} + \gamma (R_{t+2} + \gamma R_{t+3} + ...)|S_t = s]\\
	&= E[R_{t+1} + \gamma G_{t+1}|S_t = s]\\
	&= E[R_{t+1} + \gamma v(S_{t+1})|S_t = s].
	\label{bellmanv1}
\end{align}
This means that v(s) only depends on $R_{t+1}$ (the reward that can be obtained in the next step) and $\gamma v(S_{t+1})$, the discounted value function at time t+1.
What has been defined in equation \ref{bellmanv1} is known as the Bellman equation.\cite{sutton_barto}
Sutton and Barto then defines the value function given that the agent follows the policy $\pi$ as
\begin{equation}
	v_{\pi}(s) = E_{\pi}[G_t | S_t = s].
\end{equation}
Additionally, they define the action-value function (q-function),
\begin{equation}
	q_{\pi}(s,a) = E_{\pi}[G_t | S_t = s,A_t = a].
\end{equation}
The q-function is defined as the expected return of taking action a from state s and thereafter following policy $\pi$. What can be noted is that v(s) is a prediction of what the value for being in a certain state is, which is related to planning. While q(s,a) is the value for being in a certain state and taking an action, which relates to control. 

Once again, Sutton and Barto \cite{sutton_barto} decomposes the state-value and action-value functions to be in the form of the Bellman Equation in equation \ref{bellmanv1}. This results in the Bellman \textit{expectation} equations
\begin{align}
	v_{\pi}(s)	&= E_{\pi}[R_{t+1} + \gamma v_{\pi}(S_{t+1})|S_t = s]\\
	&= \sum_{a'\in A}\pi(a|s)(R^{a}_s+\gamma\sum_{s'\in S}P^{a}_{ss'}v_\pi(s'))
	\label{bellmanv2}
\end{align}
and
\begin{align}
	q_{\pi}(s,a)	&= E_{\pi}[R_{t+1} + \gamma q_{\pi}(S_{t+1},A_{t+1})|S_t = s,A_t = a]\\
	&= R^{a}_s +\gamma \sum_{s'\in S}P^{a}_{ss'}\sum_{a'\in A}\pi(a'|s')q_\pi(s',a').
	\label{bellmanq}
\end{align}
Equation \ref{bellmanv2} can be placed into vector form as:
\begin{equation}
	v_\pi = R^{\pi} + \gamma P^{\pi}v_\pi
\end{equation}
Which has the solution:
\begin{equation}
	v_\pi = (I - \gamma P^{\pi})^{-1}R^{\pi}
	\label{bellman_matrix_form}
\end{equation}
Equation \ref{bellman_matrix_form} is a concise formula which is easily implemented in code.

\section{Optimal Policies and Bellman Optimality Equations}
Sutton and Barto now defines the optimal policy, which only has value 1 if the action the agent takes maximizes $q_*(s,a)$ \cite{sutton_barto}.
This is mathematically defined as:
\begin{align}
	\pi_{*}(a|s)=\begin{cases}
		1, & \text{if a = $\argmax\limits_{a \in A}q_* (s,a)$}\\
		0, & \text{otherwise}
	\end{cases}
	\label{eq:pi_*}
\end{align}
There always exists an optimal deterministic policy for any MDP \cite{sutton_barto}. We only need to know $q_* (s,a)$ to know the optimal policy. This is important because the optimal policy describes which actions must be taken to maximize rewards, which is precisely the goal of RL.
Sutton and Barto \cite{sutton_barto} then defines what is know as the Bellman \textit{optimality} equations.
\begin{equation}
	v_*(s) = \max\limits_{a}(R^{a}_s+\gamma\sum_{s'\in S}P^{a}_{ss'}v_*(s'))
	\label{bellmanv*}
\end{equation}
and
\begin{equation}
	q_*(s,a) = R^{a}_s +\gamma \sum_{s'\in S}P^{a}_{ss'}\max\limits_{a'}q_*(s',a').
	\label{bellmanq*}
\end{equation}
What equation \ref{bellmanv*} represents is the maximum return that can be obtained from state s, when following the optimal policy $\pi_*$. Equation \ref{bellmanq*} represents the maximum return that can be obtained by taking action a state s and thereafter following the policy.
At this point, we have outlined a mathematical background for reinforcement learning problems.
In the next two chapters we will discuss methods of solving the MDP and obtaining the optimal state-value and action-value functions. 

