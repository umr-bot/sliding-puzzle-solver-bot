\graphicspath{{introduction/fig/}}

\chapter{Introduction}
\label{chap:introduction}

\section{Background}
In the modern world robots are becoming more and more part of our daily lives and society. For example the Xiaomi house cleaning robot.
Generally in robotics a manipulator (eg.
an arm) is used to manipulate an object in
the environment. It is easier to
first simulate the robots behavior in a
more simple environment which is why we will solve a similar and smaller problem as a step to a complete solution.

\section{Problem Statement}
In this project
we solve sliding puzzles using
reinforcement learning, where the same algorithm can then later be applied to the robotics problem of moving in an environment.
\section{Project Objectives}
\begin{itemize}
	\item Solve a 2x2 sliding puzzle using Reinforcement learning
	\item Solve a 3x3 puzzle using Reinforcement learning
	\item Puzzles must solve in a reasonable amount of time
\end{itemize}

\section{\color{red} Demonstration video }

\section{Scope}
Although there are many methods of solving a problem in RL, in this project we only look at two methods. These are SARSA and Q-learning. Typically for problems with large state spaces neural networks are used in conjunction with the RL methods, to save computational time. However for this project we used a certain method to overcome the state space limitation which will be discussed later.

\section{\color{red}Report Overview}