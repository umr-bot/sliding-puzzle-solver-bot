\graphicspath{{introduction/fig/}}

\chapter{Introduction}
\label{chap:introduction}

Generally in robotics a manipulator (eg.
an arm) is used to manipulate an object in
the environment. It is normally easier to
first simulate the robots behavior in a
more simple environment. In this project
we try to solve a sliding puzzle using
reinforcement learning, where the same
algorithm can then later be applied to the
robotics problem.

\section{Section heading}

This is some section with two table in it: Table~\ref{tbl:exemplars} and Table~\ref{tbl:abx_speaker}.

\begin{table}[!h]
    \mytable
    \caption{Performance of the unconstrained segmental Bayesian model on TIDigits1 over iterations in which the reference set is refined.}
    \begin{tabularx}{\linewidth}{@{}lCCCCC@{}}
        \toprule
        Metric     & 1 & 2 & 3 & 4 & 5 \\
        \midrule
        WER (\%)                        & $35.4$ & $23.5$ & $21.5$ & $21.2$ & $22.9$ \\
        Average cluster purity (\%)       & $86.5$ & $89.7$ & $89.2$ & $88.5$ & $86.6$ \\
        Word boundary $F$-score (\%)         & $70.6$ & $72.2$ & $71.8$ & $70.9$ & $69.4$ \\
        Clusters covering 90\% of data   & 20             & 13 & 13 & 13 & 13 \\
        \bottomrule
    \end{tabularx}
    \label{tbl:exemplars}
\end{table}


\begin{table}[!h]
    \renewcommand{\arraystretch}{1.1}
    \centering
    \caption{A table with an example of using multiple columns.}
    \begin{tabularx}{0.65\linewidth}{@{}lCCr@{}}
        \toprule
        & \multicolumn{2}{c}{Accuracy (\%)} \\
        \cmidrule(lr){2-3}
        Model    & Intermediate & Output & Bitrate\\
        \midrule
        Baseline & 27.5         & 26.4   & 116 \\
        VQ-VAE   & 26.0         & 22.1   & 190 \\
        CatVAE   & 28.7         & 24.3   & 215 \\
        \bottomrule
    \end{tabularx}
    \label{tbl:abx_speaker}
\end{table}

\newpage

This is a new page, showing what the page headings looks like, and showing how to refer to a figure like Figure~\ref{fig:cae_siamese}.


\begin{figure}[!t]
    \centering
%     \includegraphics[width=\linewidth]{cae_siamese}
    \includegraphics[width=0.918\linewidth]{cae_siamese}
    \caption{
    (a) The cAE as used in this chapter. The encoding layer (blue) is chosen based on performance on a development set.
    (b) The cAE with symmetrical tied weights. The encoding from the middle layer (blue) is always used.
    (c) The siamese DNN. The cosine distance between aligned frames (green and red) is either minimized or maximized depending on whether the frames belong to the same (discovered) word or not.
    }
    \label{fig:cae_siamese}
\end{figure}
